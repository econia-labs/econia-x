\documentclass[table, twocolumn]{article}
\usepackage{amsmath}
\usepackage{hyperref}
\usepackage{geometry}
\usepackage[acronym]{glossaries}
\usepackage{pgfplots}
\usepackage{xcolor}
\pgfplotsset{compat=1.18}
\usetikzlibrary{arrows.meta}
\usetikzlibrary{intersections}

% Page options.
\pagecolor{black}
\color{gray!70}
\geometry{left=35pt, top=50pt, bottom=50pt, right=35pt}

% Acronyms.
\newacronym{amm}{AMM}{Automated Market Maker}
\newacronym{cpamm}{CPAMM}{Constant Product Automated Market Maker}
\newacronym{lp}{LP}{Liquidity Provider}

% Links.
\hypersetup{colorlinks=true, allcolors={blue}}

\title{Custom Automated Market Maker}
\author{Econia Labs}
\date{}

% TikZ set for tangent lines.
% Tangent lines on graph per https://tex.stackexchange.com/a/198046.
% chktex-file 8
\makeatletter
\def\parsenode[#1]#2\pgf@nil{%
	\tikzset{label node/.style = {#1}}
\def\nodetext{#2}
}
\tikzset{%
	add node at x/.style 2 args = {%
			name path global = plot line,
			/pgfplots/execute at end plot visualization/.append = {%
					\begingroup
					\@ifnextchar[{\parsenode}{\parsenode[]}#2\pgf@nil % chktex 1
					\path[name path global = position line #1-1]
					({axis cs:#1,0}|-{rel axis cs:0,0}) --
					({axis cs:#1,0}|-{rel axis cs:0,1});
					\path[xshift = 1pt, name path global = position line #1-2]
					({axis cs:#1,0}|-{rel axis cs:0,0}) --
					({axis cs:#1,0}|-{rel axis cs:0,1});
					\path[
						name intersections =
							{of = {plot line and position line #1-1}, name = left intersection},
						name intersections =
							{of = {plot line and position line #1-2}, name = right intersection},
						label node/.append style = {pos = 1}
					] (left intersection-1) -- (right intersection-1)
					node [label node]{\nodetext};
					\endgroup
				} % chktex 9
		}
}
\makeatother


\begin{document} % chktex 17

\maketitle

\section{Spot price}\label{sec:spot-price}

The spot price of a \gls*{cpamm} is determined by the ratio of quote reserves $q$ to
base reserves $b$ per Equation~\ref{eqn:price-defined}.

\begin{equation}\label{eqn:price-defined}
	p = \frac{q}{b}
\end{equation}

This is equal to the derivative of the constant product curve for the given reserve
amounts, as illustrated in Figure~\ref{fig:spot-price}.

\begin{figure}[!htb]
	\centering
	\begin{tikzpicture}
	\begin{axis}[
			axis lines = left,
			xlabel = Base reserves,
			ylabel = Quote reserves,
			xmin = 0,
			xmax = 3,
			ymin = 0,
			ymax = 3,
			ytick=\empty,
			xtick=\empty,
			extra x ticks = {1},
			extra x tick labels = {$b$},
			extra y ticks = {1},
			extra y tick labels = {$q$},
			tick style = {thick, major tick length = 7pt},
			legend style = {fill = black, draw = gray},
			% Tangent lines on graph per https://tex.stackexchange.com/a/198046.
			tangent/.style={%
					add node at x={1}{[
									sloped, minimum width = 75pt,
									append after command =
										{(\tikzlastnode.west) edge [thick] (\tikzlastnode.east)}
								]}
				}
		]
		\addplot [
			domain = 0:5,
			samples = 100,
			color = blue,
			thick,
			tangent,
		] {1 / x};
		\node at (1.75, 1) [] {$p = -\frac{dq}{db}|_{b} = \frac{q}{b}$};
		\node at (1, 1) [circle, fill, scale = 0.5] {};
		\draw [dashed] (1, 0) -- (1, 1);
		\draw [dashed] (0, 1) -- (1, 1);
	\end{axis}
\end{tikzpicture}

	\caption{Spot price}\label{fig:spot-price}
\end{figure}

\section{Swap prices}\label{sec:swap-prices}

Since a swap involves a change in reserves, it alters the spot price. The initial spot
price $p_i$ is simply the spot price at the initial reserve amounts $b_i$ and $q_i$ per
Equation~\ref{eqn:swap-initial-price}.

\begin{equation}\label{eqn:swap-initial-price}
  p_i = \frac{q_i}{b_i}
\end{equation}

Simiarly, the final spot price $p_f$ is the spot price at the final reserve amounts
$b_f$ and $q_f$ per Equation~\ref{eqn:swap-final-price}.

\begin{equation}\label{eqn:swap-final-price}
  p_f = \frac{q_f}{b_f}
\end{equation}

Lastly, the swap price $p_s$ is the ratio of base and quote exchanged during the swap,
and is calculated differently for buys and sells.

Equation~\ref{eqn:swap-constant-product} states that the product of base and quote at
the initial and final states is constant.

\begin{equation}\label{eqn:swap-constant-product}
  b_i \cdot q_i = b_f \cdot q_f
\end{equation}

\subsection{Swap sell}\label{subsec:swap-sell}

Figure~\ref{fig:swap-sell} illustrates a swap where base input amount $b_{in}$ is sold
for quote output amount $q_{out}$.

\begin{figure}[!htb]
	\centering
	\begin{tikzpicture}
	\begin{axis}[
			axis lines = left,
			xlabel = Base reserves,
			ylabel = Quote reserves,
			xmin = 0,
			xmax = 3,
			ymin = 0,
			ymax = 3,
			ytick=\empty,
			xtick=\empty,
			extra x ticks = {0.5, 2},
			extra x tick labels = {$b_i$, $b_f$},
			extra y ticks = {2, 0.5},
			extra y tick labels = {$q_i$, $q_f$},
			tick style = {thick, major tick length = 7pt},
			legend style = {fill = black, draw = gray},
			% Tangent lines on graph per https://tex.stackexchange.com/a/198046.
			tangent/.style={add node at x={2}{[
									sloped, minimum width = 75pt,
									append after command =
										{(\tikzlastnode.west) edge [thick] (\tikzlastnode.east)}
								]},
					add node at x={0.5}{[
									sloped, minimum width = 75pt,
									append after command =
										{(\tikzlastnode.west) edge [thick] (\tikzlastnode.east)}
								]}
				}
		]
		\addplot [
			domain = 0:5,
			samples = 100,
			color = blue,
			thick,
			tangent,
		] {1 / x};
		% Initial price.
		\node at (0.325, 2.15) [] {$p_i$};
		\node at (0.5, 2) [circle, fill, scale = 0.5] {};
		% Final price.
		\node at (2.15, 0.325) [] {$p_f$};
		\node at (2, 0.5) [circle, fill, scale = 0.5] {};
		% Swap price.
		\draw [arrows = {-Latex[]}, thick] (0.5, 2) -- (2, 0.5);
		\node at (1.35, 1.35) [] {$p_s$};
		% Base delta.
		\node at (1.25, 0.25) [] {$b_{in}$};
		\draw [arrows = {-Latex[]}] (0.5, 0.125) -- (2, 0.125);
		\draw [dashed] (0.5, 0) -- (0.5, 2);
		\draw [dashed] (2, 0) -- (2, 0.5);
		% Quote delta.
		\node at (0.3, 1.25) [] {$q_{out}$};
		\draw [arrows = {-Latex[]}] (0.125, 2) -- (0.125, 0.5);
		\draw [dashed] (0, 2) -- (0.5, 2);
		\draw [dashed] (0, 0.5) -- (2, 0.5);
	\end{axis}
\end{tikzpicture}

	\caption{Swap sell}\label{fig:swap-sell}
\end{figure}

The final base and quote amounts are respectively given in
Equations~\ref{eqn:swap-sell-base-final} and \ref{eqn:swap-sell-quote-final}.

\begin{equation}\label{eqn:swap-sell-base-final}
	b_f = b_i + b_{in}
\end{equation}

\begin{equation}\label{eqn:swap-sell-quote-final}
  q_f = q_i - q_{out}
\end{equation}

The swap price $p_s$ is given in Equation~\ref{eqn:swap-sell-price}.

\begin{equation}\label{eqn:swap-sell-price}
  p_s = \frac{q_{out}}{b_{in}}
\end{equation}

Substituting Equation~\ref{eqn:swap-sell-base-final} into
Equation~\ref{eqn:swap-constant-product} yields $q_f$ as a function of $b_{in}$ and
initial reserves, $q_f(b_i, q_i, b_{in})$, per
Equation~\ref{eqn:swap-sell-quote-final-solved}.

\begin{align}\label{eqn:swap-sell-quote-final-solved}
  b_i \cdot q_i &= b_f \cdot q_f \nonumber \\
  b_i \cdot q_i &= (b_i + b_{in}) \cdot q_f \nonumber \\
  \frac{b_i \cdot q_i}{b_i + b_{in}} &= q_f \nonumber \\
  q_f(b_i, q_i, b_{in}) &= \frac{b_i \cdot q_i}{b_i + b_{in}}
\end{align}

Similarly, substituting Equation~\ref{eqn:swap-sell-quote-final-solved} into
Equation~\ref{eqn:swap-sell-quote-final} yields $q_{out}(b_i, q_i, b_{in})$
per Equation~\ref{eqn:swap-sell-quote-output}.

\begin{align}\label{eqn:swap-sell-quote-output}
  q_{out} &= q_i - q_f \nonumber \\
  q_{out} &= q_i - \frac{b_i \cdot q_i}{b_i + b_{in}} \nonumber \\
  q_{out} &= q_i \left( 1 - \frac{b_i}{b_i + b_{in}} \right) \nonumber \\
  q_{out} &= q_i \left(
    \frac{b_i + b_{in}}{b_i + b_{in}} - \frac{b_i}{b_i + b_{in}}
  \right) \nonumber \\
  q_{out} &= q_i \left( \frac{b_{in}}{b_i + b_{in}} \right) \nonumber \\
  q_{out} (b_i, q_i, b_{in}) &= \frac{q_i \cdot b_{in}}{b_i + b_{in}}
\end{align}

Lastly, subtituting Equation~\ref{eqn:swap-sell-quote-output} into
Equation~\ref{eqn:swap-sell-price} yields $p_s(b_i, q_i, b_{in})$ per
Equation~\ref{eqn:swap-sell-price-solved}.

\begin{align}\label{eqn:swap-sell-price-solved}
  p_s &= \frac{q_{out}}{b_{in}} \nonumber \\
  p_s &= \frac{q_i \cdot b_{in}}{b_i + b_{in}} \cdot \frac{1}{b_{in}} \nonumber \\
  p_s &= \frac{q_i}{b_i + b_{in}}
\end{align}

\end{document} % chktex 17
