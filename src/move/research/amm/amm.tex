\documentclass[table, twocolumn]{article}
\usepackage{amsmath}
\usepackage{hyperref}
\usepackage{geometry}
\usepackage[acronym]{glossaries}
\usepackage{pgfplots}
\usepackage{xcolor}
\pgfplotsset{compat=1.18}
\usetikzlibrary{arrows.meta}
\usetikzlibrary{intersections}

% Page options.
\pagecolor{black}
\color{gray!70}
\geometry{left=35pt, top=50pt, bottom=50pt, right=35pt}

% Acronyms.
\newacronym{amm}{AMM}{Automated Market Maker}
\newacronym{cpamm}{CPAMM}{Constant Product Automated Market Maker}
\newacronym{lp}{LP}{Liquidity Provider}

% Links.
\hypersetup{colorlinks=true, allcolors={blue}}

\title{Custom Automated Market Maker}
\author{Econia Labs}
\date{}

% TikZ set for tangent lines.
\input{figures/tangent-line-style.tex}

\begin{document} % chktex 17

\maketitle

\section{Spot price}\label{sec:spot-price}

In the absence of fees, the spot price of a \gls*{cpamm} is determined by the ratio of
quote reserves $q$ to base reserves $b$ as defined in Equation~\ref{eqn:price-defined}.
\begin{equation}\label{eqn:price-defined}
  p = \frac{q}{b}
\end{equation}

Note that this represents the instantaneous derivative of the constant product curve
for the given reserve amounts, as illustrated in Figure~\ref{fig:spot-price-no-fees}.

\begin{figure}[!htb]
	\centering
	\begin{tikzpicture}
	\begin{axis}[
			axis lines = left,
			xlabel = Base reserves,
			ylabel = Quote reserves,
			xmin = 0,
			xmax = 3,
			ymin = 0,
			ymax = 3,
			ytick=\empty,
			xtick=\empty,
			extra x ticks = {1},
			extra x tick labels = {$b$},
			extra y ticks = {1},
			extra y tick labels = {$q$},
			tick style = {thick, major tick length = 7pt},
			legend style = {fill = black, draw = gray},
			% Tangent lines on graph per https://tex.stackexchange.com/a/198046.
			tangent/.style={%
					add node at x={1}{[
									sloped, minimum width = 75pt,
									append after command =
										{(\tikzlastnode.west) edge [thick] (\tikzlastnode.east)}
								]}
				}
		]
		\addplot [
			domain = 0:5,
			samples = 100,
			color = blue,
			thick,
			tangent,
		] {1 / x};
		\node at (1.75, 1) [] {$p = -\frac{dq}{db}|_{b} = \frac{q}{b}$};
		\node at (1, 1) [circle, fill, scale = 0.5] {};
		\draw [dashed] (1, 0) -- (1, 1);
		\draw [dashed] (0, 1) -- (1, 1);
	\end{axis}
\end{tikzpicture}

	\caption{Spot price without fees}\label{fig:spot-price-no-fees}
\end{figure}

\end{document} % chktex 17
